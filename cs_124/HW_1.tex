\documentclass[11pt]{article}
% \pagestyle{empty}

\setlength{\oddsidemargin}{-0.25 in}
\setlength{\evensidemargin}{-0.25 in}
\setlength{\topmargin}{-0.9 in}
\setlength{\textwidth}{7.0 in}
\setlength{\textheight}{9.0 in}
\setlength{\headsep}{0.75 in}
\setlength{\parindent}{0.3 in}
\setlength{\parskip}{0.1 in}
\usepackage{epsf}
\usepackage{pseudocode}
\usepackage{amssymb}
\usepackage{changepage}
\usepackage{amsmath}
\usepackage{hyperref}
% \usepackage{times}
% \usepackage{mathptm}

\hypersetup{
    colorlinks=true,
    linkcolor=blue,
    filecolor=magenta,      
    urlcolor=cyan,
    pdftitle={Sharelatex Example},
    bookmarks=true,
    pdfpagemode=FullScreen,
    }

\def\O{\mathop{\smash{O}}\nolimits}
\def\o{\mathop{\smash{o}}\nolimits}
\newcommand{\e}{{\rm e}}
\newcommand{\R}{{\bf R}}
\newcommand{\Z}{{\bf Z}}

\title{CS 124}
\author{Carlos Pe\~{n}a-Lobel}
\date{January 30th 2016}
\date{Colaborated with Christian Hallas}

\begin{document}

\maketitle

\section*{Biased Die Rolls}

\begin{enumerate}

\item Suppose you are given a six-sided die, that might be biased in
an unknown way. Explain how to use die rolls to generate unbiased coin
flips, and determine the expected number of die rolls until a coin
flip is generated

\begin{adjustwidth}{.5cm}{}

    Assign probabilities $p_1, p_2, p_3, p_4, p_5, p_6$ to be $a, b, c, d, e, f$.  After 2 dice rolls, we take advantage of the symmetry of $ab$ and $ba$ or any other possible combination.  We ignore the difference between 1-4 and a 1-3 for example, and only compare whether or not the first value is greater or less than the second.  The 2 rolls will be the same with $p_{same} = a^2 +b^2+c^2+c^2+d^2+e^2+f^2$.  Now taking the infinite sum, we get an expectation of: $$\sum_{i=1}^{\infty} 2i\left(1-p_{same}\right)\left(p_{same}\right)^{i-1}.$$
    
    With an actual unbiased die, we can expect to get an unbiased coin flip after 2.4 rolls. 

\end{adjustwidth}

\item Now suppose you want to generate unbiased die
rolls (from a six-sided die) given your potentially biased die.
Explain how to do this, and again determine the expected number of
biased die rolls until an unbiased die roll is generated.

\begin{adjustwidth}{.5cm}{}
    Following a similar process as above, we now use rolls in sets of 3.  This takes advantage of the symmetry of $abc$, $acb$, $bac$, $bca$, $cab$, and $cba$.  Since each of these are equally likely we can create a map of each of these rolls to an unbiased die 1-6.  With an unbiased die, this occurs in $120/216$ of the possible sequences of 3 die rolls, or $5/9$ of the time.  Taking the same approach as before, we have a much more complicated term for when to reroll, there are $_{6} P_{1} = 6$ combos of getting all 3 as the same roll (e.g. $a^3, b^3, c^3$...), and there are $_{6} P_{2} = 30$ ways of having a number appear twice (e.g. $a^2b, a^2c, a^2d, ..., b^2a, b^2c, ... f^2d, f^2e$).  However, since we must account for order, $a^2b$ can occur in 3 ways: $a,a,b;\, a,b,a;\, b,a,a$, we must multiply each of these terms of order to be 3.  Assigning the sum of these two to be denoted as $p_{duplicate}$, we can write an expectation as: $$\sum_{i=1}^{\infty} 3i\left(1-p_{duplicate}\right)\left(p_{duplicate}\right)^{i-1}.$$
    
    Using this method with an actual unbiased die, we would expect there to be an unbiased roll after 5.4 rolls. 
    
\end{adjustwidth}

\end{enumerate}

\section*{Fibonacci}

\begin{enumerate}

\item On a platform of your choice, implement the three different
methods for computing the Fibonacci numbers (recursive, iterative, and
matrix) discussed in lecture.  Use integer variables.  How fast does
each method appear to be?  Give precise timings if possible.

\item Modify your programs so that they return the Fibonacci
numbers modulo 65536 = $2^{16}$.  For each
method, what is the largest value of $k$ such that you can compute the $k$th Fibonacci number 
(or the [$k$th Fibonacci number] modulo 65536) in one minute of machine time?  

\begin{adjustwidth}{.5cm}{}
    See code for this answer.  Python doesn't have integer integer overflow, so I ran an increasingly finer grid search overnight (when there is supposedly less background programs running) to compute the largest Fibonacci number that could be computed in one minute of machine time.  This was done for all 3 methods, using both the moded version and the unmoded version.
    
    \begin{align*}
        \text{Recursive Function: } 40 \rightarrow& 40 \\
        \text{Iterative Function: } 352,019 \rightarrow& 157,365,989 \\
        \text{Matrix Function: } 19,489,354 \rightarrow& \text{Largest number tested was $\approx10^{200}$ and ran in 40ms, see github}
    \end{align*}

    I have attached the ipython notebook file I used, a python file of the .ipynb, and here is a \href{https://github.com/cpenalobel/sportsdata/blob/master/cs_124/Homework%201.ipynb}{link to my github} with print outs of the timing: 
\end{adjustwidth}

\end{enumerate}

\section*{Run Time Notation}

\begin{enumerate}

\item Indicate for each pair of expressions $(A,B)$ in the table below
the relationship between $A$ and $B$.  Your answer should be in the
form of a table with a ``yes'' or ``no'' written in each box.  For
example, if $A$ is $O(B)$, then you should put a ``yes'' in the 
first box.

$$
\begin{array}{cc|c|c|c|c|c|}
A & B & O & o & \Omega & \omega & \Theta \\ \hline
\log{n} & \log(n^2) & \checkmark & & \checkmark & & \checkmark \\ \hline
\log (n!) & \log(n^n) & \checkmark & & \checkmark & & \checkmark \\ \hline
\sqrt[3]{n} & (\log n)^{6} & & & \checkmark & \checkmark & \\ \hline
n^22^n & 3^n & \checkmark & \checkmark & & & \\ \hline
(n^2)! & n^n & & & \checkmark & \checkmark & \\ \hline
{n^2 \over \log{n}} & n \log(n^2) & & & \checkmark & \checkmark & \\ \hline
(\log n)^{\log n} & {n \over \log(n)} & & & \checkmark & \checkmark & \\ \hline
100n + \log n & (\log n)^{3} + n & \checkmark & & \checkmark & &\checkmark  \\ \hline
\end{array}
$$

\end{enumerate}

\section*{Proofs}

For all of the problems below, when asked to give an example, you
should give a function mapping positive integers to positive integers.
(No cheating with 0's!)
\begin{enumerate}
\item Find (with proof) a function $f_1$ such that $f_1(2n)$ is $O(f_1(n))$.

\begin{adjustwidth}{.5cm}{}
Choose $f_1(n) = n$, then we have $f_1(2n) = 2n$.  By choosing $c = 3$, and N to be any positive integer (eg 1), it follows that $f_1(2n) \leq 3f_1(n)$ for any $n \geq N$.
\end{adjustwidth}

\medskip

\item Find (with proof) a function $f_2$ such that $f_2(2n)$ is not $O(f_2(n))$.

\begin{adjustwidth}{.5cm}{}
Choose $f_2(n) = 2^n$, then we have $f_2(2n) = 2^{2n}$.  Rewriting $2^{2n}$ as $2^n2^n$ or $2^nf_2(n)$ shows that in order for $f_2(2n)$ to be $O(f_2(n))$, we would need to be able to write, $2^nf_2(n) \leq cf_2(n)$.  However, $2^n$ has no upper bound, while c must be a fixed number, so for every c, it is impossible to choose an N s.t. $n \geq N$.
\end{adjustwidth}
\medskip

\item Prove that if $f(n)$ is $O(g(n))$, and $g(n)$ is $O(h(n))$, then 
$f(n)$ is $O(h(n))$.  

\begin{adjustwidth}{.5cm}{}
If $f(n)$ is $O(g(n))$, this means we can choose a $c_1$ and $N_1$ such that $\forall \, n \geq N_1, f(n) \leq c_1g(n)$.  Likewise, if $g(n)$ is $O(h(n))$, this means we can choose a $c_2$ and $N_2$ such that $\forall \, n \geq N_2, g(n) \leq c_2h(n)$.  We can now choose $N_3 = max\{N_1, N_2\}$, and since a condition is that $\forall \, n \geq N$ our equality must hold, our new $N_3$ must also hold from the inequality because it $N_3 \geq N_2, N_1$.  We can also choose $c_3 = c_1c_2$. Now we know that $f(n) \leq c_1g(n) \leq c_1c_2h(n) = c_3h(n)\,\, \forall \, n \geq N_3$, and thus $f(n)$ is $O(h(n))$. 
\end{adjustwidth}
\medskip

\item Give a proof or a counterexample:  if $f$ is not $O(g)$, then $g$ is $O(f)$.  

\begin{adjustwidth}{.5cm}{}
A counter example.  Let $f(n) = n^2$, and let $g(n) = n^2(sin(x) + 1)$. 
\end{adjustwidth}
\medskip

\item Give a proof or a counterexample:  if $f$ is $o(g)$, then $f$ is $O(g)$.

\begin{adjustwidth}{.5cm}{}
Using the definition of $o(g(n))$ we have $$\lim_{n\to\infty} \frac{f(n)}{g(n)} = 0.$$  This tells us that $\exists$ a sufficiently large N, s.t. $\forall \, n \geq N, f(n) \leq g(n)$.  This allows us to choose a constant c as 1, because this inequality is already stronger than that of the definition of $O(g(n))$.
\end{adjustwidth}

\end{enumerate}
\end{document}





